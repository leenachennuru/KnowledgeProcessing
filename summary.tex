% !TeX spellcheck = en_US

\begin{frame}{Summary}
	\begin{block}{ Two Conceptual Neighborhood Models}
		% based on the 9-intersection for binary topological relations, two models of conceptual neighborhoods among topological relations between a line and a region are developed
		
		% introduced two models for the similarity of topological line-region relations (snapshot model + smooth-transition model)
		
		\begin{enumerate}
			\item Snapshot Model
			% derives the neighborhoods by comparing pairs of topological relations and selects neighbors based on least noticeable differences
			
			\item Smooth-Transition Model
			%the smooth-transition model develops neighborhoods based on the knowledge of the deformations that may change a topological relation
		\end{enumerate}
		\textcolor{purple}{\textbf{Finding:}} \textit{Almost} identical Conceptual-Neighborhood Graphs
		% resulting similarity diagrams show some (minor) differences
	\end{block}
	\vspace{6pt}
	\begin{block}{Human-Subject Experiment}
		% tests in which human subjects were asked to organize line-region relations into groups of similar relations
		\textcolor{purple}{\textbf{Findings:}}
		\begin{itemize}
			\item confirmed that the conceptual neighborhoods identified by the two models correspond largely to the way humans conceptualize similarity about spatial relations
			\item \textbf{Finding:} groupings the subjects made indicate that the smooth-transition model captures more important aspects of the similarity of topological line-region relations than the snapshot model
			
			\item the majority of conceptual neighbors is the same in both diagrams, we conclude that the knowledge of a change process can be generally neglected when only considering topological similarity.
		\end{itemize}
	\end{block}
\end{frame}

\begin{comment}
The smooth-transition model represented the change process explicitly, whereas the snapshot model inferred change from topological differences.
\end{comment}