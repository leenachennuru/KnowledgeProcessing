% !TeX spellcheck = en_US

%TODO Sibel

\begin{frame}{Experiment}
	\begin{block}{Setup}
		\begin{itemize}
			\item 38 diagrams each of which showed a line and a region, said to be a road and a park respectively
			
			\item 2 geometrically distinct placements of the road corresponding to each of the 19 topologically distinct relations
		\end{itemize}
	\end{block}
%	For the first experiment, we produced 40 drawings of a line and a region. The region was identical in each drawing, and was bound by a thin solid line and filled with a gray tone. The line was drawn with a line weight twice that of the region boundary. The position of the line relative to the region was different in each case, and the lines were positioned so as to provide two (or, in two cases, three) geometrically distinct examples of each of the 19 topologically-distinct cases of line-region relations. Whenever possible, the line was straight in one example of each 9-intersection situation, and curved in the other(s). Subjects were told that the region was a ''park'' and that the line was a ''road'', although the representations of those features did not follow standard cartographic symbology.
	\begin{block}{Task}
		\begin{itemize}
			\item group spatial relations between line and region, road and park (parks were all the same size and shape)
			
			\item arrange the sketches into several groups, such that you would use the same verbal description for the spatial relationship between the road and the park for every sketch in each group
		\end{itemize}
	\end{block}
\end{frame}

\begin{frame}{Experiment (cont.)}
	\begin{block}{Goal}
		\begin{itemize}
			\item analyse how the subjects formed groups of similar relations
			\item check similarity with presented conceptual neighborhood models
		\end{itemize}
	\end{block}
\end{frame}
\begin{comment}
	- data obtained from an earlier human-subject study
	
	%	\begin{itemize}
	%		\item \textbf{task:} 
	%		
	%		
	%		\item Goal: Within the context of different models for conceptual neighbors, it is particularly enlightening to analyse how the subjects formed groups of similar relations.'
	%		
	%		\item 28 subjects performing tasks
	%		
	%		\item each spatial relation could be grouped as many as 112 times (4 pairs times 28 subjects) with each other relation
	%	\end{itemize}
	% As a basis for comparison, the pairs within each relation distinguished by the 9-intersection were grouped by all 28 subjects for 7 of the 19 relations, and by at least 23 subjects (82 per cent) for every relation. The maximum number of times that the stimuli in two different spatial relations were paired was 78 out of a maximum of 112 times (70 per cent).
\end{comment}

\begin{frame}{Participants}
	The 40 drawings were then printed on individual cards, and were shown to 12 native speakers of English, 12 native speakers of Chinese, 3 native speakers of German, and one native speaker of Hindi. With one exception, the instructions were given and responses were recorded in the native language of the subject. In English, 
\end{frame}

\begin{frame}{Results}
	\begin{itemize}
		\item The pairs that were neighbors by both snapshot and smooth-transition models were grouped from 0 to 78 times, with a mean of 33.6.
		
		\item Those pairs that were neighbors for smooth transitions-but not snapshots- were grouped between 0 and 66 times, with a mean of 17.3 (15.4 per cent).
		
		\item The two pairs that were snapshot neighbors-but not smooth transition neighbors- were grouped 10 and 16 times (mean = 14; 11.6 per cent).
		
		\item Perhaps most significant, however, is the fact that the 131 pairs that were neighbors by neither the snapshot model nor the smooth transitions were grouped an average of only 6.0 times by the subject (5.3 per cent of the maximum).
		
		\item Sixty pairs were never grouped by any of the 28 subjects nor any of the four possible stimulus pairs. The most frequently-grouped pair in this category was 54 times (48 per cent), but only 20 stimulus pairs with neither smooth transitions nor minimum snapshot difference were grouped 12 or more times (10 per cent of the maximum).
		
		\item 
	\end{itemize}
	
\end{frame}