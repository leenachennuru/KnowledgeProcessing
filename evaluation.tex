% !TeX spellcheck = en_US

%TODO Sibel

\begin{frame}{Experiment}
	\begin{itemize}
		\item reused data from previous human-subject experiments
		\item GOAL: Within the context of different models for conceptual neighbors, it is particularly enlightening to analyse how the subjects formed groups of similar relations.
		
		\item group spatial relations between line and region, road and park (parks were all the same size and shape)
		\item 28 subjects performing tasks
		\item 38 diagrams % each of which showed a line and a region, said to be a road and a park respectively
		2 geometrically distinct placements of the road corresponding to each of the 19 topologically distinct relations
		\ToDo Example from Mark1994
		\item each spatial relation could be grouped as many as 112 times (4 pairs times 28 subjects) with each other relation
	\end{itemize}
	% As a basis for comparison, the pairs within each relation distinguished by the 9-intersection were grouped by all 28 subjects for 7 of the 19 relations, and by at least 23 subjects (82 per cent) for every relation. The maximum number of times that the stimuli in two different spatial relations were paired was 78 out of a maximum of 112 times (70 per cent).
\end{frame}

\begin{frame}{Participants}
\end{frame}

\begin{frame}{Results}
	\begin{itemize}
		\item The pairs that were neighbors by both snapshot and smooth-transition models were grouped from 0 to 78 times, with a mean of 33.6.
		
		\item Those pairs that were neighbors for smooth transitions-but not snapshots- were grouped between 0 and 66 times, with a mean of 17.3 (15.4 per cent).
		
		\item The two pairs that were snapshot neighbors-but not smooth transition neighbors- were grouped 10 and 16 times (mean = 14; 11.6 per cent).
		
		\item Perhaps most significant, however, is the fact that the 131 pairs that were neighbors by neither the snapshot model nor the smooth transitions were grouped an average of only 6.0 times by the subject (5.3 per cent of the maximum).
		
		\item Sixty pairs were never grouped by any of the 28 subjects nor any of the four possible stimulus pairs. The most frequently-grouped pair in this category was 54 times (48 per cent), but only 20 stimulus pairs with neither smooth transitions nor minimum snapshot difference were grouped 12 or more times (10 per cent of the maximum).
		
		\item 
	\end{itemize}
	
\end{frame}